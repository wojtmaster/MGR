\documentclass[a4paper,titlepage,11pt,floatssmall]{mwrep}
\usepackage[left=2.5cm,right=2.5cm,top=2.5cm,bottom=2.5cm]{geometry}
\usepackage[OT1]{fontenc}
\usepackage{polski}
\usepackage{amsmath}
\usepackage{amsfonts}
\usepackage{amssymb}
\usepackage{graphicx}
\usepackage{float}
\usepackage{subfig}
\usepackage{url}
\usepackage{tikz}
\usetikzlibrary{arrows,calc,decorations.markings,math,arrows.meta}
\usepackage{rotating}
\usepackage[percent]{overpic}
\usepackage[utf8]{inputenc}
\usepackage{xcolor}
\usepackage{colortbl}
\usepackage{listings}
\usepackage{matlab-prettifier}
\usepackage{enumitem,amssymb}
\definecolor{szary}{rgb}{0.95,0.95,0.95}
\usepackage{siunitx}
\sisetup{detect-weight,exponent-product=\cdot,output-decimal-marker={,},per-mode=symbol,range-phrase={-},range-units=single}

%konfiguracje pakietu listings
\lstset{
  literate={ą}{{\k a}}1
           {Ą}{{\k A}}1
           {ż}{{\. z}}1
           {Ż}{{\. Z}}1
           {ź}{{\' z}}1
           {Ź}{{\' Z}}1
           {ć}{{\' c}}1
           {Ć}{{\' C}}1
           {ę}{{\k e}}1
           {Ę}{{\k E}}1
           {ó}{{\' o}}1
           {Ó}{{\' O}}1
           {ń}{{\' n}}1
           {Ń}{{\' N}}1
           {ś}{{\' s}}1
           {Ś}{{\' S}}1
           {ł}{{\l}}1
           {Ł}{{\L}}1
}
\lstset{
	backgroundcolor=\color{szary},
	frame=single,
	breaklines=true,
}
\lstdefinestyle{customlatex}{
	basicstyle=\footnotesize\ttfamily,
	%basicstyle=\small\ttfamily,
}
\lstdefinestyle{customc}{
	breaklines=true,
	frame=tb,
	language=C,
	xleftmargin=0pt,
	showstringspaces=false,
	basicstyle=\small\ttfamily,
	keywordstyle=\bfseries\color{green!40!black},
	commentstyle=\itshape\color{purple!40!black},
	identifierstyle=\color{blue},
	stringstyle=\color{orange},
}
\lstdefinestyle{custommatlab}{
	captionpos=t,
	breaklines=true,
	frame=tb,
	xleftmargin=0pt,
	language=matlab,
	showstringspaces=false,
	basicstyle=\small\ttfamily,
	%basicstyle=\scriptsize\ttfamily,
	keywordstyle=\bfseries\color{green!40!black},
	commentstyle=\itshape\color{purple!40!black},
	identifierstyle=\color{blue},
	stringstyle=\color{orange},
}
\lstdefinestyle{custompython}{
	captionpos=t,
	breaklines=true,
	frame=tb,
	xleftmargin=0pt,
	language=python,
	showstringspaces=false,
	basicstyle=\small\ttfamily,	
	keywordstyle=\bfseries\color{green!40!black},
	commentstyle=\itshape\color{purple!40!black},
	identifierstyle=\color{blue},
	stringstyle=\color{orange},
}

%wymiar tekstu (bez żywej paginy)
\textwidth 160mm \textheight 247mm

\def\figurename{Rys.}
\def\tablename{Tab.}

%konfiguracja liczby pływających elementów
\setcounter{topnumber}{0}%2
\setcounter{bottomnumber}{3}%1
\setcounter{totalnumber}{5}%3
\renewcommand{\textfraction}{0.01}%0.2
\renewcommand{\topfraction}{0.95}%0.7
\renewcommand{\bottomfraction}{0.95}%0.3
\renewcommand{\floatpagefraction}{0.35}%0.5

\begin{document}
\frenchspacing
\pagestyle{uheadings}

%strona tytułowa
\title{\bf Zastosowanie modeli statyki typu Takagi-Sugeno z następnikami hiperbolicznymi w algorytmach regulacji predykcyjnej}
\author{Wojciech Rogalski}
\date{2024}

\makeatletter
\renewcommand{\maketitle}{\begin{titlepage}
\begin{center}{\LARGE {\bf
Wydział Elektroniki i Technik Informacyjnych}}\\
\vspace{0.4cm}
{\LARGE {\bf Politechnika Warszawska}}\\
\vspace{0.3cm}
\end{center}
\vspace{5cm}
\begin{center}
{\bf \LARGE Pracownia problemowa magisterska \vskip 0.1cm}
(semestr letni 23/24L)
\end{center}
\vspace{1cm}
\begin{center}
{\bf \LARGE \@title \vskip 0.1cm}
\end{center}
\vspace{2cm}
\begin{center}
\begin{tabular}{@{}c@{\hspace{2cm}}c@{}}
\bf \Large Autor: & \bf \Large Promotor: \\
\@author & dr hab. inż. Piotr Marusak
\end{tabular}
\end{center}
\vspace*{\stretch{6}}
\begin{center}
\bf{\large{Warszawa, \@date\vskip 0.1cm}}
\end{center}
\end{titlepage}
}
\makeatother
\maketitle
\tableofcontents

\chapter{Wstęp}
Przygotowałem raport z prac w lutym. Na wstępie zaznaczę co udało się zrobić:

\begin{itemize}
\item[•] Rozmycie statyki w modelach Hammersteina i Wienera z następnikami liniowymi.
\item[•] Rozmycie statyki modelach Hammersteina i Wienera z następnikami nieliniowymi - hiperbolicznymi (sinh).
\item[•] Porównanie wyników dla przykładowych wymuszeń.
\item[•] Zaprojektowanie regulatorów DMC-analitycznego, DMC-numerycznego, DMC-SL, DMC-NPL oraz FDMC.
\item[•] Porównanie wyników otrzymanych dla poszczególnych regulatorów.
\end{itemize}

Oprócz tego starałem się otrzymać odpowiedź skokową z obiektu w pracy, natomiast tutaj sytuacja jest trochę bardziej skomplikowana i chciałbym prosić o pomoc w identyfikacji tego obiektu. Dokładne informacje w dalszej części raportu.

\chapter{Identyfikacja}
\section{Charakterystyka statyczna}
Poświęcono jej bardzo dużo uwagi, ze względu na kluczową rolę, jaką odgrywa we wspomnianych modelach Hammersteina i Wienera. Korzystając z modelu fizycznego, z równania \ref{model_fiz} wyznaczono:
\begin{equation}
\frac{dV_1}{dt} = 0 \quad \wedge \quad \frac{dV_2}{dt} = 0
\end{equation}

\noindent wobec tego:
\begin{equation}
\begin{cases}
F_1 + F_D - \alpha_1 \sqrt{h_1} &= 0 \\
\alpha_1 \sqrt{h_1} - \alpha_2 \sqrt{h_2} &= 0
\end{cases}
\end{equation}

\noindent Po prostych przekształceniach otrzymano wzór opisujący charakterystykę statyczną:
\begin{equation}
h_2 = \left( \frac{F_1 + F_D}{\alpha_2} \right)^2
\end{equation}

\noindent Wykres odpowiadający wyprowadzonemu wzorowi prezentuje się następująco:

\begin{figure}[h!]
\centering
\includegraphics[width=0.8\textwidth]{pictures/static_characteristic}
\caption{Charakterystyka statyczna $h_2(F_1)$.}
\label{static_characteristic}
\end{figure}

\noindent Założono przedział zmienności sygnału sterującego w zakresie $F_1 \in [-45, 45]$.

\newpage

\section{Wymuszenia}
Po dokonaniu pierwszego kroku identyfikacji - wykreślenia charakterystyki statycznej - uzyskano wstępne informacje o obiekcie. Równania opisujące model (\ref{model_fiz}) oraz charakterystyka statyczna przedstawiona na rys. \ref{static_characteristic} pokazuje, że obiekt jest nieliniowy, stąd dokonano jego linearyzacji w punkcie pracy, tj.:
\begin{equation}
\begin{cases}
\frac{dV_1}{dt} \cong F_1 + F_D - \alpha_1 \sqrt{\frac{V_{10}}{A}} - \frac{\alpha_1}{2 \sqrt{A \cdot V_{10}}} \cdot (V_1 - V_{10})\\
\frac{dV_2}{dt} \cong \alpha_1 \sqrt{\frac{V_{10}}{A}} - \alpha_2 \sqrt[4]{\frac{V_{20}}{C}} + \frac{\alpha_1}{2 \sqrt{A \cdot V_{10}}} \cdot (V_1 - V_{10}) - \frac{\alpha_2}{4 \sqrt[4]{C \cdot V_{20}^3}} \cdot (V_2 - V_{20})
\end{cases}
\end{equation}

\noindent Linearyzacji dokonano przyjmując jako zmienną stanu objętość cieczy w obu zbiornikach. 

\begin{equation}
x = \begin{bmatrix} V_1 & V_2 \end{bmatrix}^T
\end{equation}

Następnie, podając wygenerowaną sekwencję sygnału sterującego, zbadano rozbieżność modelu liniowego i nieliniowego.

\begin{figure}[h!]
\centering
\subfloat[Wygenerowana sekwencja sygnału sterującego $u(k)$.]{
\includegraphics[width=0.45\textwidth]{pictures/u_F1}}
\hfill
\subfloat[Sygnał wyjściowy $y(k)$.]{
\includegraphics[width=0.45\textwidth]{pictures/y_F1}}
\caption{Porównanie modelu liniowego z nieliniowym.}
\end{figure}

Otrzymano dokładnie to czego się spodziewano. Wymuszenia nie większe niż $\pm 10 \frac{cm^3}{s}$ nie powodują znacznego wytrącenia układu z położenia równowagi, dzięki czemu model liniowy bardzo dobrze aproksymuje zachowanie układu. Niestety sytuacja pogarsza się wraz z oddalaniem się od punktu pracy - model liniowy zaczyna poważnie odbiegać od modelu nieliniowego, opisującego obiekt. W celach porównawczych policzono błędy, testując model w trybie bez rekurencji (ARX) oraz z rekurencją OE, przyjmując jako kryterium jakości błąd średni kwadratowy, tj.:

\begin{equation}
E = \sum_{k=0}^N (y(k) - y^{mod}(k))^2
\end{equation}

\noindent Wcześniej dokonano podziału wygenerowanych danych dynamicznych na dwa zbiory - uczący i weryfikujący - stosując zasadę podziału $0\% - 50\% / 50\% - 100\%$. Otrzymano następujące wyniki:

\begin{description}
\item[ARX] 
\begin{equation}
E_{ucz} = \num{0.002} \hspace{1cm} E_{wer}=\num{0.003}
\end{equation}
\item[OE] 
\begin{equation}
E_{ucz} = \num{0.470} \hspace{1cm} E_{wer}=\num{0.831}
\end{equation}
\end{description}

%\newpage

\begin{figure}[p!]
\begin{center}
\Large \textbf{Model ARX}
\end{center} 
\centering
\subfloat[Zbiór uczący.]{
\includegraphics[width=0.45\textwidth]{pictures/arx_ucz}}
\hfill
\subfloat[Zbiór weryfikujący.]{
\includegraphics[width=0.45\textwidth]{pictures/arx_wer}}

\begin{center}
\Large \textbf{Model OE}
\end{center} 
\subfloat[Zbiór uczący.]{
\includegraphics[width=0.45\textwidth]{pictures/oe_ucz}}
\hfill
\subfloat[Zbiór weryfikujący.]{
\includegraphics[width=0.45\textwidth]{pictures/oe_wer}}
\caption{Symulacja odpowiednich modeli z wykorzystaniem wygenerowanej sekwencji sygnału sterującego.}
\end{figure}

\newpage

\section{Podejście inżynierskie}
Od tej pory do dalszej analizy postanowiono przyjąć model szarej skrzynki. 

\chapter{Model Hammersteina}
	\section{Następniki liniowe}
	\section{Następniki hiperboliczne}

\chapter{Model Wienera}
	\section{Następniki liniowe}
	\section{Następniki hiperboliczne}
\chapter{Podsumowanie}

\listoffigures
\listoftables
\end{document}