\documentclass[a4paper,titlepage,11pt,floatssmall]{mwrep}
\usepackage[left=2.5cm,right=2.5cm,top=2.5cm,bottom=2.5cm]{geometry}
\usepackage[OT1]{fontenc}
\usepackage{polski}
\usepackage{amsmath}
\usepackage{amsfonts}
\usepackage{amssymb}
\usepackage{graphicx}
\usepackage{float}
\usepackage{subfig}
\usepackage{url}
\usepackage{tikz}
\usetikzlibrary{arrows,calc,decorations.markings,math,arrows.meta}
\usepackage{rotating}
\usepackage[percent]{overpic}
\usepackage[utf8]{inputenc}
\usepackage{xcolor}
\usepackage{colortbl}
\usepackage{listings}
\usepackage{matlab-prettifier}
\usepackage{enumitem,amssymb}
\definecolor{szary}{rgb}{0.95,0.95,0.95}
\usepackage{siunitx}
\sisetup{detect-weight,exponent-product=\cdot,output-decimal-marker={,},per-mode=symbol,range-phrase={-},range-units=single}

%konfiguracje pakietu listings
\lstset{
  literate={ą}{{\k a}}1
           {Ą}{{\k A}}1
           {ż}{{\. z}}1
           {Ż}{{\. Z}}1
           {ź}{{\' z}}1
           {Ź}{{\' Z}}1
           {ć}{{\' c}}1
           {Ć}{{\' C}}1
           {ę}{{\k e}}1
           {Ę}{{\k E}}1
           {ó}{{\' o}}1
           {Ó}{{\' O}}1
           {ń}{{\' n}}1
           {Ń}{{\' N}}1
           {ś}{{\' s}}1
           {Ś}{{\' S}}1
           {ł}{{\l}}1
           {Ł}{{\L}}1
}
\lstset{
	backgroundcolor=\color{szary},
	frame=single,
	breaklines=true,
}
\lstdefinestyle{customlatex}{
	basicstyle=\footnotesize\ttfamily,
	%basicstyle=\small\ttfamily,
}
\lstdefinestyle{customc}{
	breaklines=true,
	frame=tb,
	language=C,
	xleftmargin=0pt,
	showstringspaces=false,
	basicstyle=\small\ttfamily,
	keywordstyle=\bfseries\color{green!40!black},
	commentstyle=\itshape\color{purple!40!black},
	identifierstyle=\color{blue},
	stringstyle=\color{orange},
}
\lstdefinestyle{custommatlab}{
	captionpos=t,
	breaklines=true,
	frame=tb,
	xleftmargin=0pt,
	language=matlab,
	showstringspaces=false,
	basicstyle=\small\ttfamily,
	%basicstyle=\scriptsize\ttfamily,
	keywordstyle=\bfseries\color{green!40!black},
	commentstyle=\itshape\color{purple!40!black},
	identifierstyle=\color{blue},
	stringstyle=\color{orange},
}
\lstdefinestyle{custompython}{
	captionpos=t,
	breaklines=true,
	frame=tb,
	xleftmargin=0pt,
	language=python,
	showstringspaces=false,
	basicstyle=\small\ttfamily,	
	keywordstyle=\bfseries\color{green!40!black},
	commentstyle=\itshape\color{purple!40!black},
	identifierstyle=\color{blue},
	stringstyle=\color{orange},
}

%wymiar tekstu (bez żywej paginy)
\textwidth 160mm \textheight 247mm

\def\figurename{Rys.}
\def\tablename{Tab.}

%konfiguracja liczby pływających elementów
\setcounter{topnumber}{0}%2
\setcounter{bottomnumber}{3}%1
\setcounter{totalnumber}{5}%3
\renewcommand{\textfraction}{0.01}%0.2
\renewcommand{\topfraction}{0.95}%0.7
\renewcommand{\bottomfraction}{0.95}%0.3
\renewcommand{\floatpagefraction}{0.35}%0.5

\begin{document}
\frenchspacing
\pagestyle{uheadings}

%strona tytułowa
\title{\bf Zastosowanie modeli statyki typu Takagi-Sugeno z następnikami hiperbolicznymi w algorytmach regulacji predykcyjnej}
\author{Wojciech Rogalski}
\date{2024}

\makeatletter
\renewcommand{\maketitle}{\begin{titlepage}
\begin{center}{\LARGE {\bf
Wydział Elektroniki i Technik Informacyjnych}}\\
\vspace{0.4cm}
{\LARGE {\bf Politechnika Warszawska}}\\
\vspace{0.3cm}
\end{center}
\vspace{5cm}
\begin{center}
{\bf \LARGE Pracownia problemowa magisterska \vskip 0.1cm}
(semestr letni 23/24L)
\end{center}
\vspace{1cm}
\begin{center}
{\bf \LARGE \@title \vskip 0.1cm}
\end{center}
\vspace{2cm}
\begin{center}
\begin{tabular}{@{}c@{\hspace{2cm}}c@{}}
\bf \Large Autor: & \bf \Large Promotor: \\
\@author & dr hab. inż. Piotr Marusak
\end{tabular}
\end{center}
\vspace*{\stretch{6}}
\begin{center}
\bf{\large{Warszawa, \@date\vskip 0.1cm}}
\end{center}
\end{titlepage}
}
\makeatother
\maketitle
\tableofcontents

\chapter{Wstęp}
Przygotowałem raport z prac w lutym. Na wstępie zaznaczę co udało się zrobić:

\begin{itemize}
\item[•] Rozmycie statyki w modelach Hammersteina i Wienera z następnikami liniowymi.
\item[•] Rozmycie statyki modelach Hammersteina i Wienera z następnikami nieliniowymi - hiperbolicznymi (sinh).
\item[•] Porównanie wyników dla przykładowych wymuszeń.
\item[•] Zaprojektowanie regulatorów DMC-analitycznego, DMC-numerycznego, DMC-SL, DMC-NPL oraz FDMC.
\item[•] Porównanie wyników otrzymanych dla poszczególnych regulatorów.
\end{itemize}

Oprócz tego starałem się otrzymać odpowiedź skokową z obiektu w pracy, natomiast tutaj sytuacja jest trochę bardziej skomplikowana i chciałbym prosić o pomoc w identyfikacji tego obiektu. Dokładne informacje w dalszej części raportu.
\chapter{Identyfikacja}
\section{Charakterystyka statyczna}
Poświęcono jej bardzo dużo uwagi, ze względu na kluczową rolę, jaką odgrywa we wspomnianych modelach Hammersteina i Wienera. Korzystając z modelu fizycznego, z równania \ref{model_fiz} wyznaczono:
\begin{equation}
\frac{dV_1}{dt} = 0 \quad \wedge \quad \frac{dV_2}{dt} = 0
\end{equation}

\noindent wobec tego:
\begin{equation}
\begin{cases}
F_1 + F_D - \alpha_1 \sqrt{h_1} &= 0 \\
\alpha_1 \sqrt{h_1} - \alpha_2 \sqrt{h_2} &= 0
\end{cases}
\end{equation}

\noindent Po prostych przekształceniach otrzymano wzór opisujący charakterystykę statyczną:
\begin{equation}
h_2 = \left( \frac{F_1 + F_D}{\alpha_2} \right)^2
\end{equation}

\noindent Wykres odpowiadający wyprowadzonemu wzorowi prezentuje się następująco:

\begin{figure}[h!]
\centering
\includegraphics[width=0.8\textwidth]{pictures/static_characteristic}
\caption{Charakterystyka statyczna $h_2(F_1)$.}
\label{static_characteristic}
\end{figure}

\noindent Założono przedział zmienności sygnału sterującego w zakresie $F_1 \in [-45, 45]$.

\newpage

\section{Wymuszenia}
Po dokonaniu pierwszego kroku identyfikacji - wykreślenia charakterystyki statycznej - uzyskano wstępne informacje o obiekcie. Równania opisujące model (\ref{model_fiz}) oraz charakterystyka statyczna przedstawiona na rys. \ref{static_characteristic} pokazuje, że obiekt jest nieliniowy, stąd dokonano jego linearyzacji w punkcie pracy, tj.:
\begin{equation}
\begin{cases}
\frac{dV_1}{dt} \cong F_1 + F_D - \alpha_1 \sqrt{\frac{V_{10}}{A}} - \frac{\alpha_1}{2 \sqrt{A \cdot V_{10}}} \cdot (V_1 - V_{10})\\
\frac{dV_2}{dt} \cong \alpha_1 \sqrt{\frac{V_{10}}{A}} - \alpha_2 \sqrt[4]{\frac{V_{20}}{C}} + \frac{\alpha_1}{2 \sqrt{A \cdot V_{10}}} \cdot (V_1 - V_{10}) - \frac{\alpha_2}{4 \sqrt[4]{C \cdot V_{20}^3}} \cdot (V_2 - V_{20})
\end{cases}
\end{equation}

\noindent Linearyzacji dokonano przyjmując jako zmienną stanu objętość cieczy w obu zbiornikach. 

\begin{equation}
x = \begin{bmatrix} V_1 & V_2 \end{bmatrix}^T
\end{equation}

Następnie, podając wygenerowaną sekwencję sygnału sterującego, zbadano rozbieżność modelu liniowego i nieliniowego.

\begin{figure}[h!]
\centering
\subfloat[Wygenerowana sekwencja sygnału sterującego $u(k)$.]{
\includegraphics[width=0.45\textwidth]{pictures/u_F1}}
\hfill
\subfloat[Sygnał wyjściowy $y(k)$.]{
\includegraphics[width=0.45\textwidth]{pictures/y_F1}}
\caption{Porównanie modelu liniowego z nieliniowym.}
\end{figure}

Otrzymano dokładnie to czego się spodziewano. Wymuszenia nie większe niż $\pm 10 \frac{cm^3}{s}$ nie powodują znacznego wytrącenia układu z położenia równowagi, dzięki czemu model liniowy bardzo dobrze aproksymuje zachowanie układu. Niestety sytuacja pogarsza się wraz z oddalaniem się od punktu pracy - model liniowy zaczyna poważnie odbiegać od modelu nieliniowego, opisującego obiekt. W celach porównawczych policzono błędy, testując model w trybie bez rekurencji (ARX) oraz z rekurencją OE, przyjmując jako kryterium jakości błąd średni kwadratowy, tj.:

\begin{equation}
E = \sum_{k=0}^N (y(k) - y^{mod}(k))^2
\end{equation}

\noindent Wcześniej dokonano podziału wygenerowanych danych dynamicznych na dwa zbiory - uczący i~weryfikujący - stosując zasadę podziału $0\% - 50\% / 50\% - 100\%$, potrzebne do późniejszego, ewentualnego dostrajania modelu. Otrzymano następujące wyniki:

\begin{description}
\item[ARX] 
\begin{equation}
E_{ucz} = \num{0.002} \hspace{1cm} E_{wer}=\num{0.003}
\end{equation}
\item[OE] 
\begin{equation}
E_{ucz} = \num{0.470} \hspace{1cm} E_{wer}=\num{0.831}
\end{equation}
\end{description}

%\newpage

\begin{figure}[p!]
\begin{center}
\Large \textbf{Model ARX}
\end{center} 
\centering
\subfloat[Zbiór uczący.]{
\includegraphics[width=0.45\textwidth]{pictures/arx_ucz}}
\hfill
\subfloat[Zbiór weryfikujący.]{
\includegraphics[width=0.45\textwidth]{pictures/arx_wer}}

\begin{center}
\Large \textbf{Model OE}
\end{center} 
\subfloat[Zbiór uczący.]{
\includegraphics[width=0.45\textwidth]{pictures/oe_ucz}}
\hfill
\subfloat[Zbiór weryfikujący.]{
\includegraphics[width=0.45\textwidth]{pictures/oe_wer}}
\caption{Symulacja odpowiednich modeli z wykorzystaniem wygenerowanej sekwencji sygnału sterującego.}
\end{figure}

\newpage

\section{Podejście inżynierskie}
Od tej pory do dalszej analizy postanowiono przyjąć model szarej skrzynki. Informacją o obiekcie był fakt, że układ był inercyjny. Zadano więc wymuszenie w postaci skoku jednostkowego i starano się aproksymować odpowiedź układu dobierając odpowiednie parametru dla modelu transmitancji \textit{First Order Plus Dead Time} (FOPDT), który wyraża się wzorem:

\begin{equation}
G(s) = \frac{K_0e^{-sT_0}}{T_1s + 1}
\end{equation}

\noindent Dobrane parametry:

\begin{equation}
K_0 = \num{0.6025} \hspace{1cm} T_0 = 100 \hspace{1cm} T_1 = 225
\end{equation}

\begin{figure}[h!]
\centering
\includegraphics[width=\textwidth]{pictures/model_fopdt}
\caption{Aproksymacja odpowiedzi skokowej układu modelem FOPDT.}
\end{figure}

\newpage

Uzyskany rezultat nie był satysfakcjonujący stąd przyjęto model \textit{Second Order Plus Dead Time} (SOPDT), tj.

\begin{equation}
G(s) = \frac{K_0e^{-sT_0}}{(T_1s + 1)(T_2s + 1)}
\end{equation}

\noindent Dobrane parametry:

\begin{equation}
K_0 = \num{0.6025} \hspace{1cm} T_0 = 100 \hspace{1cm} T_1 = 212 \hspace{1cm} T_2 = 15
\end{equation}

\noindent Wynik prezentował się następująco:

\begin{figure}[h!]
\centering
\includegraphics[width=\textwidth]{pictures/model_sopdt}
\caption{Aproksymacja odpowiedzi skokowej układu modelem SOPDT.}
\end{figure}

\newpage

Ponownie, chcąc sprawdzić skuteczność aproksymacji obiektu regulacji wygenerowanym modelem, którego równanie różnicowe jest postaci:

\begin{equation}
\begin{aligned}
y(k) = \num{1.174} y(k-1) - \num{0.2399} y(k-2) + &\num{0.02459} u_1(k-6) + \num{0.01536} u_1(k-7) \\ 
+&\num{0.02459} u_2(k-1) + \num{0.01536} u_2(k-2) 
\end{aligned}
\label{diff_eq}
\end{equation}

\noindent wygenerowano sekwencję sygnału sterującego $u_1(k)$ oraz $u_2(k)$, który są przyrostami wartości sterujących odpowiednio $F_1$ oraz $F_D$.

\begin{figure}[h!]
\begin{center}
\Large \textbf{Model ARX}
\end{center} 
\centering
\subfloat[Zbiór uczący.]{
\includegraphics[width=0.45\textwidth]{pictures/arx_ucz_sopdt}}
\hfill
\subfloat[Zbiór weryfikujący.]{
\includegraphics[width=0.45\textwidth]{pictures/arx_wer_sopdt}}

\begin{center}
\Large \textbf{Model OE}
\end{center} 
\subfloat[Zbiór uczący.]{
\includegraphics[width=0.45\textwidth]{pictures/oe_ucz_sopdt}}
\hfill
\subfloat[Zbiór weryfikujący.]{
\includegraphics[width=0.45\textwidth]{pictures/oe_wer_sopdt}}
\caption{Symulacja odpowiednich modeli z wykorzystaniem wygenerowanej sekwencji sygnału sterującego.}
\end{figure}

Błędy uznano za akceptowalne na tym poziomie identyfikacji i przyjęto wyznaczony model do dalszej analizy.
\section{Model Hammersteina}
Po przeanalizowaniu charakterystyki statycznej oraz porównaniu modelu nieliniowego z liniowym przystąpiono do identyfikacji modelu Hammersteina.
Istota polega na umieszczeniu nieliniowego bloku statycznego przed liniowym blokiem dynamicznym, co pozwala na oddzielenie nieliniowości od dynamiki systemu. Jego główną zaletą jest prostsza identyfikacja parametrów, ponieważ najpierw określa się charakterystykę statyczną, a dopiero potem analizuje dynamikę. Takie podejście lepiej odwzorowuje systemy, w których nieliniowości wynikają z właściwości aktuatorów lub czujników, a część dynamiczna pozostaje liniowa. Graficzne ujęcie opisanego modelu zilustrowano na rys. \ref{hamm_model}.

\begin{figure}[h!]
\centering
\includegraphics[width=\textwidth]{pictures/hamm_model}
\caption{Reprezentacja graficzna modelu Hammersteina.}
\label{hamm_model}
\end{figure}

\noindent Cała procedura w przypadku analizowanego obiektu wygląda następująco, sygnał sterujący jest wejściem nieliniowego bloku statycznego, którego wyjściem jest przekonwertowany sygnał $z = f(u)$. Następnie sygnał $z$ trafia do liniowego bloku dynamicznego. Dzięki temu rozdzieleniu nieliniowości i dynamiki, możliwe jest zastosowanie klasycznych metod projektowania regulatorów dla części dynamicznej, co upraszcza proces sterowania.

\subsection{Nieliniowy blok statyczny}
Nieliniowość w charakterystyce statycznej została wprowadzono za pomocą logiki rozmytej (ang. \textit{fuzzy logic}), a konkretnie za pomocą modeli rozmytych Takagi-Sugeno. Zastosowano dwa podejścia, jedno standardowe z następnikami liniowymi, natomiast drugie z następnikami hiperbolicznymi.

\subsection{Następniki liniowe}
W standardowej wersji modeli Takagi-Sugeno następniki przyjmują liniową postać, dlatego to właśnie od nich postanowiono zacząć. Rozmyto zmienną wejściową oraz wybrano odpowiednią liczbę zbiorów rozmytych. Zastosowano następniki liniowe postaci:

\begin{equation}
\begin{aligned}
\text{Reguła 1: Jeśli} \quad u(k) \quad \text{jest} \quad &U_1, \quad \text{to}: \quad y^1(k) = a_1 u(k) + b_1 \\[10pt]
\text{Reguła 2: Jeśli} \quad u(k) \quad \text{jest} \quad &U_2, \quad \text{to}: \quad y^2(k) = a_2 u(k) + b_2 \\[10pt]
&\vdots \\[10pt]
\text{Reguła 5: Jeśli} \quad u(k) \quad \text{jest} \quad &U_5, \quad \text{to}: \quad y^2(k) = a_5 u(k) + b_5 \\[10pt]
\end{aligned}
\label{nastepniki_lin}
\end{equation}

\noindent Natomiast wyjście systemu rozmytego obliczano zgodnie ze wzorem \ref{wniosek}.

\newpage

\begin{figure}[h!]
\centering
\includegraphics[width=\textwidth]{pictures/hamm_linearFis}
\caption{Zbiory rozmyte - następniki liniowe.}
\end{figure}

Zarówno do budowy modelu, jak i wyznaczenia parametrów następników wykorzystano narzędzia oferowane przez MATLAB w ramach \textit{Fuzzy Logic Toolbox}. Korzystając z funkcji \verb+sugfis()+ zbudowano nieliniowy model rozmyty typu Takagi - Sugeno. Zdecydowano się na pięć zbiorów rozmytych o gaussowskim kształcie, co zapewnia różniczkowalność (SZAU). Następnie, dzięki wykorzystaniu \verb+addInput()+, \verb+addOutput()+, \verb+addMF()+ udało się zbudować bazę reguł - \verb+addRule()+, co bezpośrednio przełożyło się na wyznaczenie współczynników pierwszej iteracji. Konieczne było późniejsze ręczne dostrajanie modelu, które przy względnie dużej liczbie zbiorów nie przysporzyło dużo problemów. Ostatecznie zdefiniowano następujące wartości parametrów następników:

\begin{table}[h!]
\centering
\renewcommand{\arraystretch}{1.2} % Zwiększa wysokość wierszy
\begin{tabular}{|>{\centering\arraybackslash}m{3cm}|>{\centering\arraybackslash}m{3cm}|>{\centering\arraybackslash}m{3cm}|}
\hline
Nr reguły & Współczynnik $a_r$ & Współczynnik $b_r$ \\ \hline
1 & $\num{0.7895}$ & $\num{0.0001}$ \\ \hline
2 & $\num{0.8982}$ & $\num{0.0002}$ \\ \hline
3 & $\num{1.0933}$ & $\num{0.0001}$ \\ \hline
4 & $\num{1.0489}$ & $\num{0}$ \\ \hline
5 & $\num{1.2034}$ & $\num{0.0001}$ \\ \hline
\end{tabular}
\end{table}

\newpage

\subsection{Następniki nieliniowe}
Wprowadzając następniki w postaci hiperbolicznej spodziewano się zachowania dokładności przy jednoczesnym zmniejszeniu liczby zbiorów rozmytych [Robust observer-based controller design for Takagi–Sugeno systems with nonlinear consequent parts]. Sformułowano następującą bazę reguł:

\begin{equation}
\begin{aligned}
\text{Reguła 1: Jeśli} \quad u(k) \quad \text{jest} \quad &U_1, \quad \text{to}: \quad y^1(k) = a_1 \sinh\left(\frac{u(k)}{b_1}\right) \\[10pt]
\text{Reguła 2: Jeśli} \quad u(k) \quad \text{jest} \quad &U_2, \quad \text{to}: \quad y^2(k) = a_2 \sinh\left(\frac{u(k)}{b_2}\right) \\[10pt]
\text{Reguła 3: Jeśli} \quad u(k) \quad \text{jest} \quad &U_3, \quad \text{to}: \quad y^2(k) = a_3 \sinh\left(\frac{u(k)}{b_3}\right)
\end{aligned}
\label{nastepniki_nlin}
\end{equation}

Zgodnie z oczekiwaniami udało się wprowadzić mniejszą liczbę zbiorów rozmytych.

\begin{figure}[h!]
\centering
\includegraphics[width=\textwidth]{pictures/hamm_nonlinearFis}
\caption{Zbiory rozmyte - następniki nieliniowe.}
\end{figure}

\newpage

Procedura dostrajania parametrów następników różniła się w stosunku do odpowiedników liniowych. Wynikało to z tego, że wykorzystywane narzędzie do budowy modelu rozmytego domyślnie stroi parametry dla następników liniowych, stąd konieczność znacznej modyfikacji i ręcznego dostrajania. Następnie, po ręcznym dostrojeniu wykorzytsano funkcję z pakietu \textit{Optimization Toolbox}, mianowicie \verb+fminsearch()+. Wybrano taką kolejność ze względu na fakt, że odwracając kolejność, tzn. stosując najpierw metodę Neldera-Meada, otrzymywane wyniki były gorsze niż te wybrane ręcznie. Było to spowodowane skłonnością do wpadania algorytmu w minima lokalne. Istotnym aspektem w tym przypadku okazała się normalizacja argumentu, bowiem kształt funkcji $\sinh()$ zmienia się bardzo gwałtownie dla rosnących wartości zmiennej (rys. {\ref{sinh}}. 

\begin{figure}[h!]
\centering
\includegraphics[width=\textwidth]{pictures/sinh}
\caption{Wykres funkcji $\sinh()$.}
\label{sinh}
\end{figure}

Ostateczne wartości współczynników następników reguł zebrano w tab. \ref{nonlinear_coeff}.

\begin{table}[h!]
\centering
\renewcommand{\arraystretch}{1.2} % Zwiększa wysokość wierszy
\begin{tabular}{|>{\centering\arraybackslash}m{2cm}|>{\centering\arraybackslash}m{3cm}|>{\centering\arraybackslash}m{3cm}|}
\hline
Nr reguły & Współczynnik $a_r$ & Współczynnik $b_r$ \\ \hline
1. & $\num{46.4941}$ & $\num{62.8862}$ \\ \hline
2. & $\num{36.2347}$ & $\num{36.2592}$ \\ \hline
3. & $\num{110.4841}$ & $\num{98.7690}$ \\ \hline    
\end{tabular}
\caption{Współczynniki hiperbolicznych następników reguł.}
\label{nonlinear_coeff}
\end{table}

\newpage

\subsection{Porównanie}
Dostroiwszy oba modele przyszedł czas na ich porównanie. Wygenerowano pięć sekwencji losowo zmieniającego się sygnału sterującego, następnie wynik został porównany do rezultatu uzyskanego dla modelu nieliniowego - wzorcowego - obliczonego za pomocą zmodyfikowanej metody Eulera. Wskaźnikiem porównawczym był błąd średnio kwadratowy. Aby zaznaczyć jakie korzyści wnosi nieliniowa część modelu Hammersteina na dokładność modelowania, w każdej sekwencji obliczono także wskaźnik dla modelu liniowego.

\begin{figure}[h!]
\centering
\subfloat[Następniki liniowe]{
\includegraphics[width=0.7\textwidth]{pictures/HammersteinLinearModel_1}}
\vspace{0.5cm}
\subfloat[Następniki nieliniowe]{
\includegraphics[width=0.7\textwidth]{pictures/HammersteinNonlinearModel_1}}
\caption{Porównanie modelu Hammersteina z następnikami liniowymi i nieliniowymi - pierwsza sekwencja.}
\label{first_hamm}
\end{figure}

\begin{figure}[p]
\centering
\subfloat[Następniki liniowe]{
\includegraphics[width=0.75\textwidth]{pictures/HammersteinLinearModel_2}}
\vspace{0.5cm}
\subfloat[Następniki nieliniowe]{
\includegraphics[width=0.75\textwidth]{pictures/HammersteinNonlinearModel_2}}
\caption{Porównanie modelu Hammersteina z następnikami liniowymi i nieliniowymi - druga sekwencja.}
\end{figure}

\begin{figure}[p]
\centering
\subfloat[Następniki liniowe]{
\includegraphics[width=0.75\textwidth]{pictures/HammersteinLinearModel_3}}
\vspace{0.5cm}
\subfloat[Następniki nieliniowe]{
\includegraphics[width=0.75\textwidth]{pictures/HammersteinNonlinearModel_3}}
\caption{Porównanie modelu Hammersteina z następnikami liniowymi i nieliniowymi - trzecia sekwencja.}
\end{figure}

\begin{figure}[p]
\centering
\subfloat[Następniki liniowe]{
\includegraphics[width=0.75\textwidth]{pictures/HammersteinLinearModel_4}}
\vspace{0.5cm}
\subfloat[Następniki nieliniowe]{
\includegraphics[width=0.75\textwidth]{pictures/HammersteinNonlinearModel_4}}
\caption{Porównanie modelu Hammersteina z następnikami liniowymi i nieliniowymi - czwarta sekwencja.}
\end{figure}

\begin{figure}[p]
\centering
\subfloat[Następniki liniowe]{
\includegraphics[width=0.75\textwidth]{pictures/HammersteinLinearModel_5}}
\vspace{0.5cm}
\subfloat[Następniki nieliniowe]{
\includegraphics[width=0.75\textwidth]{pictures/HammersteinNonlinearModel_5}}
\caption{Porównanie modelu Hammersteina z następnikami liniowymi i nieliniowymi - piąta sekwencja.}
\label{last_hamm}
\end{figure}

\newpage

Zaprezentowane wykresy na rys. \ref{first_hamm} - \ref{last_hamm} ilustrują przede wszystkim zysk zastosowania modelu Hammersteina do opisu obiektu. Otrzymane wyniki są nieporównywalnie lepsze w stosunku do modelu liniowego. Natomiast przyglądając się z bliska, można zauważyć również poprawę dzięki wprowadzeniu dodatkowej nieliniowości w definicji następników rozmytego modelu typu Takagi-Sugeno. W tab. \ref{comparison_hamm} zebrano uzyskane wyniki.

\begin{table}[h!]
\centering
\renewcommand{\arraystretch}{1.2}
\begin{tabular}{|>{\centering\arraybackslash}m{2cm}|>{\centering\arraybackslash}m{3cm}|>{\centering\arraybackslash}m{3cm}|>{\centering\arraybackslash}m{3cm}|}
\hline
\multirow{2}{*}{Nr sekwencji} & \multirow{2}{*}{Model liniowy} & \multicolumn{2}{c|}{Model Hammersteina} \\ \cline{3-4}
 &  & Następniki liniowe & Następniki nieliniowe \\ \hline
1. & $\num{4.062}$ & $\num{0.395}$ & $\num{0.348}$ \\ \hline
2. & $\num{2.701}$ & $\num{0.210}$ & $\num{0.177}$ \\ \hline
3. & $\num{1.283}$ & $\num{0.146}$ & $\num{0.129}$ \\ \hline
4. & $\num{3.893}$ & $\num{0.597}$ & $\num{0.528}$ \\ \hline
5. & $\num{0.600}$ & $\num{0.070}$ & $\num{0.030}$ \\ \hline
\end{tabular}
\caption{Porównanie modeli.}
\label{comparison_hamm}
\end{table}

Należy pamiętać, że w przypadku następników hiperbolicznych zredukowano liczbę zbiorów rozmytych. Zatem udało się poprawić dokładność modelowania, przy jednoczesnym zmniejszeniu liczby reguł modelu TS.
\chapter{Model Wienera}
Istota modelu Wienera jest dokładnie taka sama jak w przypadku modelu Hammersteina z tym, że nieliniową statykę poprzedzono liniową dynamika - odwrotnie niż jak to było w przypadku omówionego modelu.

\begin{figure}[h!]
\centering
\includegraphics[width=\textwidth]{pictures/wien_model}
\caption{Reprezentacja graficzna modelu Wienera.}
\end{figure}

Sygnał wejściowy trafia na liniowy blok dynamiczny, gdzie jest przekonwertowany na sygnał $v = f(u)$, który trafia na nieliniową statykę, której wyjściem jest sygnał $y$. W przypadku dynamiki skorzystano z wyznaczonego wcześniej modelu dynamicznego (\ref{diff_eq}), natomiast zmianie uległa charakterystyka statyczna.

\section{Następniki liniowe}
W przypadku modelu Wienera aproksymacja charakterystyki statycznej przysporzyła więcej problemów niżeli w przypadku modelu Hammersteina. Przyjęty przedział rozmywania zmiennej $v \in [-25;25]$ podzielono na pięć zbiorów.

\begin{figure}[h!]
\centering
\includegraphics[width=0.6\textwidth]{pictures/fuzzy_set_wien}
\caption{Zbiory rozmyte - następniki liniowe.}
\end{figure}

\newpage

\noindent Dzięki zastosowanemu podziałowi udało się otrzymać następującą aproksymację charakterystyki statycznej.

\begin{figure}[h!]
\centering
\subfloat[Zbiór uczący.]{
\includegraphics[width=0.45\textwidth]{pictures/static_char_wien_lin_ucz}}
\hfill
\subfloat[Zbiór weryfikujący]{
\includegraphics[width=0.45\textwidth]{pictures/static_char_wien_lin_wer}}
\caption{Aproksymacja charakterystyki statycznej przez model rozmyty - następniki liniowe.}
\end{figure}

Następniki liniowe reguł przybrały postać zaprezentowaną w \ref{nastepniki_lin}. Ponownie pierwsze przybliżenie zostało wyznaczone w za pomocą funkcji MATLAB, tym razem jednak niezbędne okazało się ręczne dostrajanie otrzymanych współczynników. Do problemu zdecydowano podejść w następujący sposób - starano się minimalizować błąd zbioru uczącego, następnie analizowano zachowanie modelu w przypadku zbiory weryfikującego. Wygenerowano kilka sekwencji sygnału sterującego i porównano zachowanie modelu dynamicznego i modelu Wienera.

\newpage

%%%%%%%%%%%%%%%%%%%%%% PIERWSZA SEKWENCJA %%%%%%%%%%%%%%%%%%%%%%
\begin{figure}[h!]

\begin{center}
\Large \textbf{I sekwencja} \\
\vspace{0.5cm}
\Large \textbf{Model dynamiczny}
\end{center}

\centering
\subfloat[Zbiór uczący.]{
\includegraphics[width=0.45\textwidth]{pictures/arx_ucz_11}}
\hfill
\subfloat[Zbiór weryfikujący]{
\includegraphics[width=0.45\textwidth]{pictures/arx_wer_11}}

\begin{center}
\Large \textbf{Model Wienera}
\end{center}

\subfloat[Zbiór uczący.]{
\includegraphics[width=0.45\textwidth]{pictures/arx_wien_ucz_11}}
\hfill
\subfloat[Zbiór weryfikujący]{
\includegraphics[width=0.45\textwidth]{pictures/arx_wien_wer_11}}
\caption{Porównanie przebiegu sygnału wyjściowego modelu dynamicznego i modelu Wienera w trybie bez rekurencji.}
\end{figure}

Jak widać na powyższych rysunkach dokładność osiągnięta przez model Wienera w trybie ARX jest nieakceptowalna - postanowiono ręcznie dostroić model. Dokonano tego tylko poprzez zmianę wartości współczynników lokalnych modeli systemu rozmytego. Uzyskane rezultaty zaprezentowano na rys. \ref{wien_arx}.

\newpage

\begin{figure}[h!]
\centering
\subfloat[Zbiór uczący.]{
\includegraphics[width=0.7\textwidth]{pictures/arx_wien_ucz_12}}
\vfill
\subfloat[Zbiór weryfikujący]{
\includegraphics[width=0.7\textwidth]{pictures/arx_wien_wer_12}}
\caption{Przebiegu sygnału wyjściowego modelu Wienera w trybie bez rekurencji po dostrojeniu współczynników następników.}
\label{wien_arx}
\end{figure}

Model Wienera po ręcznym strojeniu osiągnął próg akceptowalności w kontekście kryterium jakości - $E = \num{0.1}$. Analogiczne podejście zastosowano w przypadku testowania modelu w trybie rekurencyjnym.

\newpage

\begin{figure}[h!]

\begin{center}
\Large \textbf{I sekwencja} \\
\vspace{0.5cm}
\Large \textbf{Model dynamiczny}
\end{center}

\centering
\subfloat[Zbiór uczący.]{
\includegraphics[width=0.45\textwidth]{pictures/oe_ucz_11}}
\hfill
\subfloat[Zbiór weryfikujący]{
\includegraphics[width=0.45\textwidth]{pictures/oe_wer_11}}

\begin{center}
\Large \textbf{Model Wienera}
\end{center}

\subfloat[Zbiór uczący.]{
\includegraphics[width=0.45\textwidth]{pictures/oe_wien_ucz_11}}
\hfill
\subfloat[Zbiór weryfikujący]{
\includegraphics[width=0.45\textwidth]{pictures/oe_wien_wer_11}}
\caption{Porównanie przebiegu sygnału wyjściowego modelu dynamicznego i modelu Wienera w trybie rekurencyjnym.}
\end{figure}

Model Wienera w trybie rekurencyjnym osiągnął bardzo duże wartości błędów. Natomiast postanowiono przyjąć inną strategię dostrajania, widać bowiem, że przemnożenie wyjścia modelu przez pewną stałą dałoby już zadowalający rezultat. Konieczne okazało się również dostrojenie wartości poszczególnych współczynników modeli lokalnych systemu rozmytego, aby osiągnąć zakładaną dokładność. Efekt przedstawiono na rys. \ref{wien_oe}.

\newpage

\begin{figure}[h!]
\centering
\subfloat[Zbiór uczący.]{
\includegraphics[width=0.7\textwidth]{pictures/oe_wien_ucz_12}}
\vfill
\subfloat[Zbiór weryfikujący]{
\includegraphics[width=0.7\textwidth]{pictures/oe_wien_wer_12}}
\caption{Przebiegu sygnału wyjściowego modelu Wienera w trybie rekurencyjnym po dostrojeniu współczynników następników.}
\label{wien_oe}
\end{figure}

Jak pokazują powyższe ilustracje osiągnięto satysfakcjonującą dokładność, zbliżoną do tych uzyskanych podczas symulacji modelu Hammersteina (w niektórych przypadkach nawet lepszą). Dostrojony model poddano testom, generując kolejne dwie sekwencje sygnału sterującego.

%%%%%%%%%%%%%%%%%%%%%% DRUGA SEKWENCJA %%%%%%%%%%%%%%%%%%%%%%

\begin{figure}[p!]

\begin{center}
\Large \textbf{II sekwencja} \\
\vspace{0.5cm}
\Large \textbf{Model dynamiczny}
\end{center}

\centering
\subfloat[Zbiór uczący.]{
\includegraphics[width=0.45\textwidth]{pictures/arx_ucz_22}}
\hfill
\subfloat[Zbiór weryfikujący.]{
\includegraphics[width=0.45\textwidth]{pictures/arx_wer_22}}

\begin{center}
\Large \textbf{Model Wienera}
\end{center}

\subfloat[Zbiór uczący.]{
\includegraphics[width=0.45\textwidth]{pictures/arx_wien_ucz_22}}
\hfill
\subfloat[Zbiór weryfikujący.]{
\includegraphics[width=0.45\textwidth]{pictures/arx_wien_wer_22}}
\caption{Porównanie przebiegu sygnału wyjściowego modelu dynamicznego i modelu Wienera w trybie bez rekurencji.}
\end{figure}

\begin{figure}[p!]

\begin{center}
\Large \textbf{II sekwencja} \\
\vspace{0.5cm}
\Large \textbf{Model dynamiczny}
\end{center}

\centering
\subfloat[Zbiór uczący.]{
\includegraphics[width=0.45\textwidth]{pictures/oe_ucz_22}}
\hfill
\subfloat[Zbiór weryfikujący.]{
\includegraphics[width=0.45\textwidth]{pictures/oe_wer_22}}

\begin{center}
\Large \textbf{Model Wienera}
\end{center}

\subfloat[Zbiór uczący.]{
\includegraphics[width=0.45\textwidth]{pictures/oe_wien_ucz_22}}
\hfill
\subfloat[Zbiór weryfikujący.]{
\includegraphics[width=0.45\textwidth]{pictures/oe_wien_wer_22}}
\caption{Porównanie przebiegu sygnału wyjściowego modelu dynamicznego i modelu Wienera w trybie rekurencyjnym.}
\end{figure}

%%%%%%%%%%%%%%%%%%%%%% TRZECIA SEKWENCJA %%%%%%%%%%%%%%%%%%%%%%

\begin{figure}[p!]

\begin{center}
\Large \textbf{III sekwencja} \\
\vspace{0.5cm}
\Large \textbf{Model dynamiczny}
\end{center}

\centering
\subfloat[Zbiór uczący.]{
\includegraphics[width=0.45\textwidth]{pictures/arx_ucz_33}}
\hfill
\subfloat[Zbiór weryfikujący.]{
\includegraphics[width=0.45\textwidth]{pictures/arx_wer_33}}

\begin{center}
\Large \textbf{Model Wienera}
\end{center}

\subfloat[Zbiór uczący.]{
\includegraphics[width=0.45\textwidth]{pictures/arx_wien_ucz_33}}
\hfill
\subfloat[Zbiór weryfikujący.]{
\includegraphics[width=0.45\textwidth]{pictures/arx_wien_wer_33}}
\caption{Porównanie przebiegu sygnału wyjściowego modelu dynamicznego i modelu Wienera w trybie bez rekurencji.}
\end{figure}

\newpage

\begin{figure}[h!]

\begin{center}
\Large \textbf{III sekwencja} \\
\vspace{0.5cm}
\Large \textbf{Model dynamiczny}
\end{center}

\centering
\subfloat[Zbiór uczący.]{
\includegraphics[width=0.45\textwidth]{pictures/oe_ucz_33}}
\hfill
\subfloat[Zbiór weryfikujący.]{
\includegraphics[width=0.45\textwidth]{pictures/oe_wer_33}}

\begin{center}
\Large \textbf{Model Wienera}
\end{center}

\subfloat[Zbiór uczący.]{
\includegraphics[width=0.45\textwidth]{pictures/oe_wien_ucz_33}}
\hfill
\subfloat[Zbiór weryfikujący.]{
\includegraphics[width=0.45\textwidth]{pictures/oe_wien_wer_33}}
\caption{Porównanie przebiegu sygnału wyjściowego modelu dynamicznego i modelu Wienera w trybie rekurencyjnym.}
\end{figure}

Na podstawie powyższych wykresów, można wysnuć wniosek, że model Wienera został dobrze dostrojony, ponieważ w każdym z testowanych przypadków uzyskano błąd mniejszy niż przyjęty za graniczny $E = \num{0.1}$. Co ciekawe, strojąc model dobrano pewny współczynnik, który przemnażał wyjście liniowej dynamiki $v = f(u)$. Najlepszy rezultat uzyskano, gdy ta stała była równa wzmocnieniu statycznemu otrzymanej transmitancji, tj.

\begin{equation}
K_{stat} = \lim_{z \to 1} G(z)
\end{equation}

Wcześniej w równaniu różnicowymi zadbano, aby wzmocnienie statyczne wynosiło $1$, jednak mimo to, konieczne okazało się zastosowanie współczynnika regulującego. 

\newpage

\section{Następniki hiperboliczne}
W przypadku następników hiperbolicznych postąpiono analogicznie jak w przypadku modelu Hammersteina - zredukowano liczbę zbiorów przynależności jednocześnie starając się zachować dokładność osiągniętą dla następników liniowych. Z doświadczeń empirycznych postanowiono przyjąć taką samą postać następników jak poprzednio, tj.:

\begin{equation}
\text{Reguła n: Jeśli} \quad u^n(k) \quad \text{jest} \quad U_n, \quad \text{to}: \quad y^n(k) = a_n \cdot \sinh(b_n u^n(k) + c_n)
\end{equation}

Natomiast zbiory przynależności zmiennej $v$ prezentowały się następująco:

\begin{figure}[h!]
\centering
\includegraphics[width=0.6\textwidth]{pictures/fuzzy_set_wien_nlin}
\caption{Zbiory rozmyte - następniki hiperboliczne.}
\end{figure}

Z kolei dokładność aproksymacji charakterystyki statycznej przedstawiono na rys. \ref{static_char_wien_nlin}.

\begin{figure}[h!]
\centering
\subfloat[Zbiór uczący.]{
\includegraphics[width=0.45\textwidth]{pictures/static_char_wien_nlin_ucz}}
\hfill
\subfloat[Zbiór weryfikujący.]{
\includegraphics[width=0.45\textwidth]{pictures/static_char_wien_nlin_wer}}
\caption{Aproksymacja charakterystyki statycznej przez model rozmyty - następniki hiperboliczne.}
\label{static_char_wien_nlin}
\end{figure}

\newpage

Przystąpiono do testowania modeli, generując kolejne sekwencje sygnału sterującego.

%%%%%%%%%%%%%%%%%%%%%% PIERWSZA SEKWENCJA %%%%%%%%%%%%%%%%%%%%%%
\begin{figure}[h!]

\begin{center}
\Large \textbf{I sekwencja} \\
\vspace{0.5cm}
\Large \textbf{Model dynamiczny}
\end{center}

\centering
\subfloat[Zbiór uczący.]{
\includegraphics[width=0.45\textwidth]{pictures/arx_ucz_41}}
\hfill
\subfloat[Zbiór weryfikujący]{
\includegraphics[width=0.45\textwidth]{pictures/arx_wer_41}}

\begin{center}
\Large \textbf{Model Wienera}
\end{center}

\subfloat[Zbiór uczący.]{
\includegraphics[width=0.45\textwidth]{pictures/arx_wien_ucz_41}}
\hfill
\subfloat[Zbiór weryfikujący]{
\includegraphics[width=0.45\textwidth]{pictures/arx_wien_wer_41}}
\caption{Porównanie przebiegu sygnału wyjściowego modelu dynamicznego i modelu Wienera w trybie bez rekurencji.}
\end{figure}

Po raz kolejny model Wienera wymagał ręcznego strojenia. Tym razem natomiast oprócz lokalnych poprawek modeli systemu rozmytego zdecydowano się na przemnożenie wyjścia modelu przez stałą. 

\newpage

\begin{figure}[h!]
\centering
\subfloat[Zbiór uczący.]{
\includegraphics[width=0.7\textwidth]{pictures/arx_wien_ucz_42}}
\hfill
\subfloat[Zbiór weryfikujący]{
\includegraphics[width=0.7\textwidth]{pictures/arx_wien_wer_42}}
\caption{Przebiegi sygnału wyjściowego modelu Wienera w trybie bez rekurencji po dostrojeniu współczynników następników.}
\end{figure}

Udało się uzyskać satysfakcjonujące rezultaty. Model Wienera w trybie ARX po dostrojeniu prezentował się bardzo dobrze.

\newpage

\begin{figure}[h!]

\begin{center}
\Large \textbf{I sekwencja} \\
\vspace{0.5cm}
\Large \textbf{Model dynamiczny}
\end{center}

\centering
\subfloat[Zbiór uczący.]{
\includegraphics[width=0.45\textwidth]{pictures/oe_ucz_41}}
\hfill
\subfloat[Zbiór weryfikujący]{
\includegraphics[width=0.45\textwidth]{pictures/oe_wer_41}}

\begin{center}
\Large \textbf{Model Wienera}
\end{center}

\subfloat[Zbiór uczący.]{
\includegraphics[width=0.45\textwidth]{pictures/oe_wien_ucz_41}}
\hfill
\subfloat[Zbiór weryfikujący]{
\includegraphics[width=0.45\textwidth]{pictures/oe_wien_wer_41}}
\caption{Porównanie przebiegu sygnału wyjściowego modelu dynamicznego i modelu Wienera w trybie rekurencyjnym.}
\end{figure}

Model testowany w trybie rekurencyjnym wymagał dużej ingerencji. Przede wszystkim przemnożenie wyjścia modelu przez stałą - w tym przypadku $\num{0.54}$ - znacznie poprawiło otrzymane wyniki. Dostrojenie modeli lokalnych pozwoliło uzyskać satysfakcjonujący rezultat.

\newpage

\begin{figure}[h!]
\centering
\subfloat[Zbiór uczący.]{
\includegraphics[width=0.7\textwidth]{pictures/oe_wien_ucz_42}}
\hfill
\subfloat[Zbiór weryfikujący]{
\includegraphics[width=0.7\textwidth]{pictures/oe_wien_wer_42}}
\caption{Przebiegi sygnału wyjściowego modelu Wienera w trybie rekurencyjnym po dostrojeniu współczynników następników.}
\end{figure}

Z tak wystrojony model rozmyty poddano kolejnym dwóm testom, generując odpowiednią sekwencję sygnału sterującego.

%%%%%%%%%%%%%%%%%%%%%% DRUGA SEKWENCJA %%%%%%%%%%%%%%%%%%%%%%
\begin{figure}[p!]

\begin{center}
\Large \textbf{II sekwencja} \\
\vspace{0.5cm}
\Large \textbf{Model dynamiczny}
\end{center}

\centering
\subfloat[Zbiór uczący.]{
\includegraphics[width=0.45\textwidth]{pictures/arx_ucz_55}}
\hfill
\subfloat[Zbiór weryfikujący]{
\includegraphics[width=0.45\textwidth]{pictures/arx_wer_55}}

\begin{center}
\Large \textbf{Model Wienera}
\end{center}

\subfloat[Zbiór uczący.]{
\includegraphics[width=0.45\textwidth]{pictures/arx_wien_ucz_55}}
\hfill
\subfloat[Zbiór weryfikujący]{
\includegraphics[width=0.45\textwidth]{pictures/arx_wien_wer_55}}
\caption{Porównanie przebiegu sygnału wyjściowego modelu dynamicznego i modelu Wienera w trybie bez rekurencji.}
\end{figure}

\begin{figure}[p!]

\begin{center}
\Large \textbf{II sekwencja} \\
\vspace{0.5cm}
\Large \textbf{Model dynamiczny}
\end{center}

\centering
\subfloat[Zbiór uczący.]{
\includegraphics[width=0.45\textwidth]{pictures/oe_ucz_55}}
\hfill
\subfloat[Zbiór weryfikujący]{
\includegraphics[width=0.45\textwidth]{pictures/oe_wer_55}}

\begin{center}
\Large \textbf{Model Wienera}
\end{center}

\subfloat[Zbiór uczący.]{
\includegraphics[width=0.45\textwidth]{pictures/oe_wien_ucz_55}}
\hfill
\subfloat[Zbiór weryfikujący]{
\includegraphics[width=0.45\textwidth]{pictures/oe_wien_wer_55}}
\caption{Porównanie przebiegu sygnału wyjściowego modelu dynamicznego i modelu Wienera w trybie rekurencyjnym.}
\end{figure}

%%%%%%%%%%%%%%%%%%%%%% TRZECIA SEKWENCJA %%%%%%%%%%%%%%%%%%%%%%
\begin{figure}[p!]

\begin{center}
\Large \textbf{III sekwencja} \\
\vspace{0.5cm}
\Large \textbf{Model dynamiczny}
\end{center}

\centering
\subfloat[Zbiór uczący.]{
\includegraphics[width=0.45\textwidth]{pictures/arx_ucz_66}}
\hfill
\subfloat[Zbiór weryfikujący]{
\includegraphics[width=0.45\textwidth]{pictures/arx_wer_66}}

\begin{center}
\Large \textbf{Model Wienera}
\end{center}

\subfloat[Zbiór uczący.]{
\includegraphics[width=0.45\textwidth]{pictures/arx_wien_ucz_66}}
\hfill
\subfloat[Zbiór weryfikujący]{
\includegraphics[width=0.45\textwidth]{pictures/arx_wien_wer_66}}
\caption{Porównanie przebiegu sygnału wyjściowego modelu dynamicznego i modelu Wienera w trybie bez rekurencji.}
\end{figure}

\newpage

\begin{figure}[h!]

\begin{center}
\Large \textbf{III sekwencja} \\
\vspace{0.5cm}
\Large \textbf{Model dynamiczny}
\end{center}

\centering
\subfloat[Zbiór uczący.]{
\includegraphics[width=0.45\textwidth]{pictures/oe_ucz_66}}
\hfill
\subfloat[Zbiór weryfikujący]{
\includegraphics[width=0.45\textwidth]{pictures/oe_wer_66}}

\begin{center}
\Large \textbf{Model Wienera}
\end{center}

\subfloat[Zbiór uczący.]{
\includegraphics[width=0.45\textwidth]{pictures/oe_wien_ucz_66}}
\hfill
\subfloat[Zbiór weryfikujący]{
\includegraphics[width=0.45\textwidth]{pictures/oe_wien_wer_66}}
\caption{Porównanie przebiegu sygnału wyjściowego modelu dynamicznego i modelu Wienera w trybie rekurencyjnym.}
\end{figure}

Na podstawie powyższych wykresów można wysnuć wniosek, że udało się osiągnąć zamierzony cel - zredukowano liczbę zbiorów rozmytych i tym samym modeli lokalnych przy zachowaniu wcześniej osiągniętej dokładności. Niemniej, ręczne dostrajanie współczynników następników hiperbolicznych było procesem czasochłonnym i mało intuicyjnym, bowiem chcąc poprawić działanie modelu w danym przedziale wartości często okazywało się, że należy zmienić parametry modelu, który w tym przedziale wydawał się być nieistotny. Ostateczny efekt uznano za satysfakcjonujący. 
\chapter{Podsumowanie}
Na obecnym etapie prac udało się zgromadzić część narzędzi potrzebnych do osiągnięcia zamierzonego celu pracy. Udało się między innymi:

\begin{itemize}
\item[•] rozmyć model obiektu zbiorników w kaskadzie
\item[•] opracować modele Hammersteina i Wienera dla rozmytego modelu zbiorników - zarówno z następnikami liniowymi, jak i nieliniowymi
\item[•] zaimplementować algorytm DMC w wersji analitycznej, numerycznej z ograniczeniami oraz z uwzględnieniem zakłóceń
\end{itemize}

Najbliższy plan pracy zakłada:
\begin{itemize}
\item[•] opracowanie algorytmu DMC z sukcesywną linearyzacją oraz nieliniową predykcją
\item[•] zebranie odpowiedzi skokowych z rzeczywistego obiektu i przeprowadzenie testów
\end{itemize}

%\begin{table}[h!]
%\centering
%\begin{tabular}{>{\centering\arraybackslash}p{3cm}|>{\centering\arraybackslash}p{3cm}|>{\centering\arraybackslash}p{3cm}|>{\centering\arraybackslash}p{3cm}|>{\centering\arraybackslash}p{3cm}}
% & \multicolumn{2}{>{\centering\arraybackslash}p{6cm}|}{Model dynamiczny} &  \multicolumn{2}{>{\centering\arraybackslash}p{6cm}}{Model Wienera} \\ \hline
% & ARX & OE & ARX & OE \\ \hline
%$I$ sekwencja & & & & \\
%$II$ sekwencja & & & & \\
%$III$ sekwencja & & & & 
%\end{tabular}
%\caption{Błędy zbioru uczącego.}
%\end{table}
%
%\begin{table}[h!]
%\centering
%\begin{tabular}{>{\centering\arraybackslash}p{3cm}|>{\centering\arraybackslash}p{3cm}|>{\centering\arraybackslash}p{3cm}|>{\centering\arraybackslash}p{3cm}|>{\centering\arraybackslash}p{3cm}}
% & \multicolumn{2}{>{\centering\arraybackslash}p{6cm}|}{Model dynamiczny} &  \multicolumn{2}{>{\centering\arraybackslash}p{6cm}}{Model Wienera} \\ \hline
% & ARX & OE & ARX & OE \\ \hline
%$I$ sekwencja & & & & \\
%$II$ sekwencja & & & & \\
%$III$ sekwencja & & & & 
%\end{tabular}
%\caption{Błędy zbioru weryfikującego.}
%\end{table}

\listoffigures
%\listoftables
\end{document}