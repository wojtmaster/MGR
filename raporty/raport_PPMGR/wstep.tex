\chapter{Wstęp}
Praca zawiera porównanie modeli Hammersteina oraz Wienera w regulacji kaskadowej. Bazą porównania był obiekt opisany równaniami fizycznymi postaci:
\begin{equation}
\begin{cases}
\frac{dV_1}{dt} = F_1 + F_D - F_2(h_1) \\
\frac{dV_2}{dt} = F_2(h_1) - F_3(h_2) \\
F_2(h_1) = \alpha_1 \sqrt{h_1}, \quad F_3(h_2) = \alpha_2 \sqrt{h_2}, \quad V_1(h_1) = A_1h_1, \quad V_2(h_2) = C_2h_2^2, \quad F_1(t) = F_{1in}(t-\tau)  
\end{cases}
\label{model_fiz}
\end{equation}

\begin{itemize}
\item[•] Stałe: 
\begin{equation}
A_1 = 540cm^2, \quad C_2 = \num{0.85}, \quad \alpha_1 = 26, \quad \alpha_2 = 20
\end{equation}

\item[•] Punkt pracy:
\begin{equation}
F_1 = 90 \frac{cm^3}{s}, \quad F_D = 30 \frac{cm^3}{s}, \quad \tau = 100, \quad h_2 = 36cm
\end{equation}
\end{itemize}

\noindent gdzie użyte oznaczenia odpowiadają tym zastosowanym na rys. \ref{schemat}.

\begin{figure}[h!]
\centering
\includegraphics[width=0.8\textwidth]{pictures/schemat}
\caption{Obiekt regulacji automatycznej.}
\label{schemat}
\end{figure}

Wartością sterującą był dopływ $F_{1in}$ natomiast zakłóceniem - $F_D$. Z kolei wyjściem - wartością regulowaną - wysokość cieczy w drugim zbiorniku $h_2$. W pierwszej kolejności dokonano identyfikacji modelu, sprawdzono jego nieliniowość i dobrano odpowiedni rząd dynamiki modelu liniowego.