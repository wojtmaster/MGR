\chapter{Cel pracy}
Celem niniejszej pracy jest analiza porównawcza modeli Hammersteina i Wienera w modelach statyki typu Takagi - Sugeno z następnikami hiperbolicznymi w algorytmach regulacji predykcyjnej. Jednym z rozważanych obiektów sterowania były zbiorniki w kaskadzie z dokładnym opisem matematycznym, do którego identyfikacji wykorzystano model hybrydowy, natomiast drugim jest obiekt rzeczywisty, którego budowa i zasada działania została dokładnie ujęta w \cite{20} - identyfikacja z wykorzystaniem modelu empirycznego [Odwołanie do MODI, PODA]. Praca w zależności od obiektu obejmowała różne zagadnienia. W przypadku zbiorników pracowano na danych wygenerowanych, natomiast w przypadku instalacji CO2 - danych zebranych. Jednakże do najważniejszych filarów pracy należała:
\begin{itemize}
\item[•] identyfikację modelu
\item[•] rozdzielenie procesu na część dynamiczną oraz statyczną
\item[•] zebranie odpowiedzi skokowych
\item[•] strojenie regulatora predykcyjnego
\item[•] porównanie jakości regulacji z wykorzystaniem modeli Hammersteina i Wienera
\item[•] porównanie jakości regulacji z wykorzystaniem hiperbolicznych następników w modelach Takagi - Sugeno
\end{itemize}

W kwestii algorytmów regulacji predykcyjnej wybrano regulator DMC (\textit{Dynamic Matrix Control}) w różnych wariantach:
\begin{itemize}
\item[•] klasyczny liniowy algorytm DMC
\item[•] rozmyty algorytm DMC
\item[•] rozmyty algorytm DMC z sukcesywną linearyzacją 
\item[•] rozmyty algorytm DMC z nieliniowa predykcja i linearyzacja
\end{itemize}

Wszystkie wymienione algorytmy zostały zaimplementowane na obiekcie symulacyjnym - zbiorniki w kaskadzie - a najlepszy z pośród nich został zaimplementowany również na obiekcie rzeczywistym.s