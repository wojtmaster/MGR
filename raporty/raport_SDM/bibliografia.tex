\begin{thebibliography}{99}
\setstretch{1.5} % Ustawienie interlinii na 1.5

\bibitem{10} A. Janczak. \textit{Identification of Nonlinear Systems Using Neural Networks and Polynomial Models.} Zielona Góra, Springer, 2005.

\bibitem{20} L. Ljung. \textit{System Identification. Theory for the User,} wyd. 2. Szwecja, Prentice Hall, 1999.

\bibitem{21} L. Ljung. \textit{Above: A Harrunerstein model. Below: A Wiener model}, [ilustracja]. \\ W: \textit{System Identification. Theory for the User,} wyd. 2. Szwecja, Prentice Hall, 1999. 

\bibitem{150} M. Ławryńczuk. \textit{Modelowanie i identyfikacja}. Warszawa, 2022.

\bibitem{160} M. Ławryńczuk. \textit{Sterowanie procesów ciągłych}. Warszawa, 2022.

\bibitem{170} M. Ławryńczuk, P. Marusak. \textit{Sztuczna inteligencja w automatyce}. Warszawa, 2009–2018.

\bibitem{171} M. Ławryńczuk, P. Marusak. \textit{Regulator typu Takagi–Sugeno otrzymany za pomoca podejscia PDC}, [ilustracja]. W: \textit{Sztuczna inteligencja w automatyce}. Warszawa, 2009–2018.

\bibitem{30} J. M. Maciejowski. \textit{Predictive Control with Constraints.} Harlow, Prentice Hall, 2002.

\bibitem{35} K. Malinowski, P. Tatjewski. \textit{Podstawy Automatyki}, wyd. 2. Warszawa, 2016.

\bibitem{40} P. Marusak. \textit{Regulacja predykcyjna obiektów nieliniowych z zastosowaniem techniki DMC i modelowania rozmytego}. Warszawa, Wydział Elektroniki i Technik Informacyjnych, 2002.

\bibitem{50} P. Marusak, J. Pułaczewski, P. Tatjewski. \textit{Algorytmy DMC z uwzględnieniem ograniczeń sterowania}, vol. 1.  Opole, 1999.

\bibitem{60} P. Marusak, P. Tatjewski. \textit{Fuzzy Dynamic Matrix Control algorithms for nonlinear plants}, vol. 2. Międzyzdroje, 2000.

\bibitem{70} K. Mehran. \textit{Takagi-Sugeno Fuzzy Modeling for Process Control}. Newcastle, 2008.

\bibitem{80} H. Moodi, M. Farrokhi. \textit{Robust observer-based controller design for Takagi–Sugeno systems with nonlinear consequent parts}. W: \textit{Fuzzy Sets and Systems}. Amsterdam, Elsevier B.V., 2015.\\ Nr 273, s. 141-154, ISNN 0165-0114.

\bibitem{90} K. Rykaczewski. \textit{Systemy rozmyte i ich zastosowania}. Toruń, 2006.

\bibitem{120} P. Tatjewski. \textit{Sterowanie zaawansowane obiektów przemysłowych. Struktury i algorytmy}.\\ Warszawa, Akademicka Oficyna Wydawnicza EXIT, 2016.

\bibitem{121} P. Tatjewski. \textit{Warstwowa struktura sterowania obiektem przemysłowym}, [ilustracja].\\ W: \textit{Sterowanie zaawansowane obiektów przemysłowych. Struktury i algorytmy}.\\ Warszawa, Akademicka Oficyna Wydawnicza EXIT, 2016.

\bibitem{122} P. Tatjewski. \textit{Zasada regulacji predykcyjnej}, [ilustracja].\\ W: \textit{Sterowanie zaawansowane obiektów przemysłowych. Struktury i algorytmy}.\\ Warszawa, Akademicka Oficyna Wydawnicza EXIT, 2016.

\newpage

\bibitem{123} P. Tatjewski. \textit{Przykład odpowiedzi wyjscia obiektu y na skok sterowania u}, [ilustracja].\\ W: \textit{Sterowanie zaawansowane obiektów przemysłowych. Struktury i algorytmy}.\\ Warszawa, Akademicka Oficyna Wydawnicza EXIT, 2016.

\bibitem{130} A. Piegat. \textit{Modelowanie i sterowanie rozmyte}.\\ Warszawa, Akademicka Oficyna Wydawnicza EXIT, 1999.

\end{thebibliography}