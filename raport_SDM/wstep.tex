\chapter{Wstęp}
Przedmiotem rozważań niniejszej pracy było wykorzystanie hiperbolicznych następników w modelach statyki typu Takagi - Sugeno w algorytmach regulacji predykcyjnej. Temat ten zatem porusza kilka istotnych zagadnień: regulacja predykcyjna, modelowanie rozmyte typu \\ Takagi - Sugno, modele Hammerteina i Wienera.

Początki regulacji predykcyjnej datuje się na lata 70 XX. wieku. Wtedy też rozwój techniki mikroprocesorowej umożliwił wdrażanie - bardziej kosztownych obliczeniowo niż bardzo dobrze znany regulator PID - algorytmów regulacji predykcyjnej zaliczonych wówczas do grupy tzw. zaawansowanych technik regulacji (\textit{advanced control techniques}) \cite{35}. Na początku popularnością cieszyły się głównie w branży chemicznej i petrochemicznej, ale jak pokazują czasy współczesne, zyskały uznanie również w takich branżach jak: robotyka, motoryzacja czy energetyka. Zawdzięczają to zaletom jakie oferują \cite{160, 170, 120}:
\begin{itemize}
\item możliwość uwzględnienia ograniczeń sygnałów zarówno wejściowych, jak i wyjściowych
\item dobra jakość regulacji także w przypadku obiektów o trudnej dynamice - duże opóźnienia
\item efektywne regulowanie procesu wielowymiarowego
\item możliwość zaprojektowania nieliniowego algorytmy regulacji predykcyjnej 
\item możliwość uwzględnienia wpływu zakłóceń - jeśli dostępny jest ich mierzalny model 
\item główna zasada regulacji predykcyjnej jest stosunkowo prosta i uniwersalna
\end{itemize}

Omawiając kolejne zagadnienie nie sposób nie wspomnieć o regulacji nieliniowej, która pod koniec lat 70. ubiegłego stulecia wyznaczyła kierunek rozwoju przemysłowych układów regulacji i sterowania. Istotą regulacji nieliniowej bowiem jest projektowanie regulatora od razu dla obiektu modelowanego w szerokim zakresie zmian wielkości wejściowych i wyjściowych, a nie jak w przypadku regulatorów liniowych - tylko w otoczeniu wybranego punktu pracy. Takie podejście sprawia, że regulator jest odporny, zatem działa stabilnie przy zmianach parametrów obiektu równocześnie zapewniając odpowiednią jakość regulacji. Szybko okazało się, że podejście wykorzystujące zbiory rozmyte (\textit{fuzzy sets}) oraz logikę rozmytą (\textit{fuzzy logic}) mogą być solidnym i skutecznym narzędziem szczególnie tam, gdzie opis analityczny obiektu nie jest w pełni dostępny. Na szczególne wyróżnienie zasługują modele o strukturze Takagi-Sugeno, które bardzo dobrze nadają się do opisu silnie nieliniowych obiektów regulacji. Modelowanie rozmyte, zaraz obok modelowania z wykorzystaniem sieci neuronowych było podejściem najintensywniej rozwijanym i praktycznie stosowanym od lat 90. ubiegłego wieku \cite{160, 120}. 

Ostatnim kluczowym zagadnieniem poruszanym w pracy są modele Hammersteina i
Wienera. Opracowane w pierwszej połowie XX wieku lecz dynamiczne rozwijane i praktycznie stosowane dopiero w drugiej połowie, co było związane z postępem w technologii systemów i modelowaniu nieliniowym. Ich istotę docenił autor w \cite{20} wskazując, że owe modele są efektywnym podejściem do modelowania systemów, które mają liniową dynamikę, ale wprowadzają nieliniowości statyczne na wejściu (model Hammersteina) lub wyjściu (model Wienera).