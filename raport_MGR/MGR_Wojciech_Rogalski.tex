\documentclass[a4paper,twoside,titlepage,11pt,floatssmall]{mwrep}
\usepackage[inner=3.0cm,outer=2.0cm,top=2.5cm,bottom=2.5cm]{geometry}
\usepackage[OT1]{fontenc}
\usepackage{polski}
\usepackage{amsmath}
\usepackage{amsfonts}
\usepackage{amssymb}
\usepackage{graphicx}
\usepackage{float}
\usepackage{subfig}
\usepackage{url}
\usepackage{tikz}
\usetikzlibrary{arrows,calc,decorations.markings,math,arrows.meta}
\usepackage{rotating}
\usepackage[percent]{overpic}
\usepackage[utf8]{inputenc}
\usepackage{xcolor}
\usepackage{colortbl}
\usepackage{listings}
\usepackage{matlab-prettifier}
\usepackage{enumitem,amssymb}
\definecolor{szary}{rgb}{0.95,0.95,0.95}
\usepackage{siunitx}
\sisetup{detect-weight,exponent-product=\cdot,output-decimal-marker={,},per-mode=symbol,range-phrase={-},range-units=single}

%konfiguracje pakietu listings
\lstset{
  literate={ą}{{\k a}}1
           {Ą}{{\k A}}1
           {ż}{{\. z}}1
           {Ż}{{\. Z}}1
           {ź}{{\' z}}1
           {Ź}{{\' Z}}1
           {ć}{{\' c}}1
           {Ć}{{\' C}}1
           {ę}{{\k e}}1
           {Ę}{{\k E}}1
           {ó}{{\' o}}1
           {Ó}{{\' O}}1
           {ń}{{\' n}}1
           {Ń}{{\' N}}1
           {ś}{{\' s}}1
           {Ś}{{\' S}}1
           {ł}{{\l}}1
           {Ł}{{\L}}1
}
\lstset{
	backgroundcolor=\color{szary},
	frame=single,
	breaklines=true,
}
\lstdefinestyle{customlatex}{
	basicstyle=\footnotesize\ttfamily,
	%basicstyle=\small\ttfamily,
}
\lstdefinestyle{customc}{
	breaklines=true,
	frame=tb,
	language=C,
	xleftmargin=0pt,
	showstringspaces=false,
	basicstyle=\small\ttfamily,
	keywordstyle=\bfseries\color{green!40!black},
	commentstyle=\itshape\color{purple!40!black},
	identifierstyle=\color{blue},
	stringstyle=\color{orange},
}
\lstdefinestyle{custommatlab}{
	captionpos=t,
	breaklines=true,
	frame=tb,
	xleftmargin=0pt,
	language=matlab,
	showstringspaces=false,
	basicstyle=\small\ttfamily,
	%basicstyle=\scriptsize\ttfamily,
	keywordstyle=\bfseries\color{green!40!black},
	commentstyle=\itshape\color{purple!40!black},
	identifierstyle=\color{blue},
	stringstyle=\color{orange},
}
\lstdefinestyle{custompython}{
	captionpos=t,
	breaklines=true,
	frame=tb,
	xleftmargin=0pt,
	language=python,
	showstringspaces=false,
	basicstyle=\small\ttfamily,	
	keywordstyle=\bfseries\color{green!40!black},
	commentstyle=\itshape\color{purple!40!black},
	identifierstyle=\color{blue},
	stringstyle=\color{orange},
}

%wymiar tekstu (bez żywej paginy)
\textwidth 160mm \textheight 247mm

\def\figurename{Rys.}
\def\tablename{Tab.}

%konfiguracja liczby pływających elementów
\setcounter{topnumber}{0}%2
\setcounter{bottomnumber}{3}%1
\setcounter{totalnumber}{5}%3
\renewcommand{\textfraction}{0.01}%0.2
\renewcommand{\topfraction}{0.95}%0.7
\renewcommand{\bottomfraction}{0.95}%0.3
\renewcommand{\floatpagefraction}{0.35}%0.5

\begin{document}
\frenchspacing
\pagestyle{uheadings}

%strona tytułowa
\title{\bf Zastosowanie modeli statyki typu Takagi-Sugeno z następnikami hiperbolicznymi w algorytmach regulacji predykcyjnej}
\author{Wojciech Rogalski}
\date{2024}

\makeatletter
\renewcommand{\maketitle}{\begin{titlepage}
\begin{center}{\LARGE {\bf
Wydział Elektroniki i Technik Informacyjnych}}\\
\vspace{0.4cm}
{\LARGE {\bf Politechnika Warszawska}}\\
\vspace{0.3cm}
\end{center}
\vspace{5cm}
\begin{center}
{\bf \LARGE Pracownia problemowa magisterska \vskip 0.1cm}
(semestr zimowy 24/25Z)
\end{center}
\vspace{1cm}
\begin{center}
{\bf \LARGE \@title \vskip 0.1cm}
\end{center}
\vspace{2cm}
\begin{center}
\begin{tabular}{@{}c@{\hspace{2cm}}c@{}}
\bf \Large Autor: & \bf \Large Promotor: \\
\@author & dr hab. inż. Piotr Marusak
\end{tabular}
\end{center}
\vspace*{\stretch{6}}
\begin{center}
\bf{\large{Warszawa, \@date\vskip 0.1cm}}
\end{center}
\end{titlepage}
}
\makeatother
\maketitle
\tableofcontents

\chapter{Wstęp}
\begin{thebibliography}{99}
\setstretch{1.5} % Ustawienie interlinii na 1.5

\bibitem{10} A. Janczak. \textit{Identification of Nonlinear Systems Using Neural Networks and Polynomial Models.} Zielona Góra, Springer, 2005.

\bibitem{20} L. Ljung. \textit{System Identification. Theory for the User,} wyd. 2. Szwecja, Prentice Hall, 1999.

\bibitem{21} L. Ljung. \textit{Above: A Harrunerstein model. Below: A Wiener model}, [ilustracja]. \\ W: \textit{System Identification. Theory for the User,} wyd. 2. Szwecja, Prentice Hall, 1999. 

\bibitem{150} M. Ławryńczuk. \textit{Modelowanie i identyfikacja}. Warszawa, 2022.

\bibitem{160} M. Ławryńczuk. \textit{Sterowanie procesów ciągłych}. Warszawa, 2022.

\bibitem{170} M. Ławryńczuk, P. Marusak. \textit{Sztuczna inteligencja w automatyce}. Warszawa, 2009–2018.

\bibitem{171} M. Ławryńczuk, P. Marusak. \textit{Regulator typu Takagi–Sugeno otrzymany za pomoca podejscia PDC}, [ilustracja]. W: \textit{Sztuczna inteligencja w automatyce}. Warszawa, 2009–2018.

\bibitem{30} J. M. Maciejowski. \textit{Predictive Control with Constraints.} Harlow, Prentice Hall, 2002.

\bibitem{35} K. Malinowski, P. Tatjewski. \textit{Podstawy Automatyki}, wyd. 2. Warszawa, 2016.

\bibitem{40} P. Marusak. \textit{Regulacja predykcyjna obiektów nieliniowych z zastosowaniem techniki DMC i modelowania rozmytego}. Warszawa, Wydział Elektroniki i Technik Informacyjnych, 2002.

\bibitem{50} P. Marusak, J. Pułaczewski, P. Tatjewski. \textit{Algorytmy DMC z uwzględnieniem ograniczeń sterowania}, vol. 1.  Opole, 1999.

\bibitem{60} P. Marusak, P. Tatjewski. \textit{Fuzzy Dynamic Matrix Control algorithms for nonlinear plants}, vol. 2. Międzyzdroje, 2000.

\bibitem{70} K. Mehran. \textit{Takagi-Sugeno Fuzzy Modeling for Process Control}. Newcastle, 2008.

\bibitem{80} H. Moodi, M. Farrokhi. \textit{Robust observer-based controller design for Takagi–Sugeno systems with nonlinear consequent parts}. W: \textit{Fuzzy Sets and Systems}. Amsterdam, Elsevier B.V., 2015.\\ Nr 273, s. 141-154, ISNN 0165-0114.

\bibitem{90} K. Rykaczewski. \textit{Systemy rozmyte i ich zastosowania}. Toruń, 2006.

\bibitem{120} P. Tatjewski. \textit{Sterowanie zaawansowane obiektów przemysłowych. Struktury i algorytmy}.\\ Warszawa, Akademicka Oficyna Wydawnicza EXIT, 2016.

\bibitem{121} P. Tatjewski. \textit{Warstwowa struktura sterowania obiektem przemysłowym}, [ilustracja].\\ W: \textit{Sterowanie zaawansowane obiektów przemysłowych. Struktury i algorytmy}.\\ Warszawa, Akademicka Oficyna Wydawnicza EXIT, 2016.

\bibitem{122} P. Tatjewski. \textit{Zasada regulacji predykcyjnej}, [ilustracja].\\ W: \textit{Sterowanie zaawansowane obiektów przemysłowych. Struktury i algorytmy}.\\ Warszawa, Akademicka Oficyna Wydawnicza EXIT, 2016.

\newpage

\bibitem{123} P. Tatjewski. \textit{Przykład odpowiedzi wyjscia obiektu y na skok sterowania u}, [ilustracja].\\ W: \textit{Sterowanie zaawansowane obiektów przemysłowych. Struktury i algorytmy}.\\ Warszawa, Akademicka Oficyna Wydawnicza EXIT, 2016.

\bibitem{130} A. Piegat. \textit{Modelowanie i sterowanie rozmyte}.\\ Warszawa, Akademicka Oficyna Wydawnicza EXIT, 1999.

\end{thebibliography}

\listoffigures
\listoftables
\end{document}